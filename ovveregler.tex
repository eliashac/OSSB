%%%%%%%%%%%%%%%%%%%%%%%%%%%%%%%%%%%%%%%%%%%%%%%%%%%%%%%%%%%%%%%%%%%%%%%%%%%%%%%%
% Ovveregler
%%%%%%%%%%%%%%%%%%%%%%%%%%%%%%%%%%%%%%%%%%%%%%%%%%%%%%%%%%%%%%%%%%%%%%%%%%%%%%%%
\settowidth{\versewidth}{Bakom brödbutiken bodde Baskerbosses båda bröder,}

%\poemtitle{Ovveregler}

%------------------------------------------------

\begin{enumerate}[leftmargin=1cm,labelindent=16pt,label=\bfseries \arabic*.]
\item Sektionsmärket bör sitta över hjärtat för att visa sin kärlek till OPEN.
\item Dryckesmärken bör sitta på vänster ben för att visa att OPENisten står stadigt.
\item Gasquemärken bör sitta på höger ben för att visa att OPENisten stadigt står.
\item Namn eller smeknamn bör finnas redovisat på en godtyckklig del av ovven, höger ben rekomenderas.
\item Tvagning bör endast ske om OPENisten är i overallen. 
Detta gäller även torktumling.
\item En sann OPENist hedrar den ärorika taditionen att bära sin vackert röda overall den tredje onsdagen varje månad då detta är Ovve-Onsdag.
\item Ovveoskulden får endast tas med tänderna och när vederbörande person har på sig ovven.
\end{enumerate}

%----------------------------------------------------------------------------------------