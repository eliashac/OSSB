% Här skall du sätta in en genomsnittligt lång rad från sången. Detta för att veta hur texten ska centreras under kompileringen.
\settowidth{\versewidth}{I Spanien hör det till god ton och etikett}

% Sångtitel, här skriver du in sångtiteln
\poemtitle{Skål för vattnet!}

%------------------------------------------------

\begin{verse}[\versewidth]

% Här skriver du in vilken melodi som används (glöm inte att ta med "Mel."!)
\flagverse{}
\emph{Mel. Rule Britannia/Arne}\\!

%------------------------------------------------

% Flagverse numrerar verserna. Om sången innehåller refränger så hänvisar jag till det format som används i internationalen.tex
% Se även pdf:en för vägledning
% \\ avslutar en rad, * förhindrar sidbrytning efter den raden
% ! markerar styckesslut (obs. det får bli sidbrytning efter styckesslut)
\flagverse{1.}
I Spanien hör det till god ton och etikett\\*
att gå omkring och skryta vitt och brett.\\*
De trodde att de skulle slå oss lätt!\\*
Men vi är modesta.\\*
Ja, vår profil är låg\\*
Men vi är de som väger tyngst på havets våg!\\*
Skål för vattnet, tjoho för H20!\\*
Skål för ön där bara blyga britter bo!\\*
Skål för vattnet som gör vår flotta stark,\\*
hell allt vatten som ger flyt åt vår monark!\\!



%------------------------------------------------

% Obligatorisk
\end{verse}

%----------------------------------------------------------------------------------------