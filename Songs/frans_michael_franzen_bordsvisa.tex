%%%%%%%%%%%%%%%%%%%%%%%%%%%%%%%%%%%%%%%%%%%%%%%%%%%%%%%%%%%%%%%%%%%%%%%%%%%%%%%%
% Frans Michael Franzén Bordsvisa
%%%%%%%%%%%%%%%%%%%%%%%%%%%%%%%%%%%%%%%%%%%%%%%%%%%%%%%%%%%%%%%%%%%%%%%%%%%%%%%%

\settowidth{\versewidth}{När skämtet tar ordet vid vänskapens bord}

\poemtitle{Frans Michael Franzén Bordsvisa}

%------------------------------------------------

\begin{verse}[\versewidth]

%------------------------------------------------

\flagverse{1.}
När skämtet tar ordet vid vänskapens bord\\*
med fingret åt glasen, som dofta,\\*
så drick och var glad: på vår sorgliga jord\\*
man gläder sig aldrig för ofta.\\*
En blomma är glädjen: i dag slår hon ut,\\*
i morgon förvissnar hon redan,\\*
just nu, då du kan, hav en lycklig minut\\*
och tänk på den kommande sedan.\\!

%------------------------------------------------

\flagverse{2.}
Vem drog ej en suck över tidernas lopp?\\*
Dock sitt ej och dröm på kalaset!\\*
Här lev i sekunden, och hela ditt hopp\\*
se fyllas och tömmas - i glaset!\\*
Här sörj ej för glaset: om fullt, så drick ut;\\*
om tomt, så försänd det att fyllas;\\*
och minns, att det sköna och goda förut,\\*
sen glädjen och nöjet, må hyllas.\\!

\end{verse}

%----------------------------------------------------------------------------------------