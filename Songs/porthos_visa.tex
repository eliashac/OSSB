%%%%%%%%%%%%%%%%%%%%%%%%%%%%%%%%%%%%%%%%%%%%%%%%%%%%%%%%%%%%%%%%%%%%%%%%%%%%%%%%
% Porthos visa
%%%%%%%%%%%%%%%%%%%%%%%%%%%%%%%%%%%%%%%%%%%%%%%%%%%%%%%%%%%%%%%%%%%%%%%%%%%%%%%%

\settowidth{\versewidth}{Nej, för fan bara blunda och svälj}

\poemtitle{Porthos visa}

%------------------------------------------------

\begin{verse}[\versewidth]

\flagverse{}
\emph{Mel. You can't get a man with a gun}\\!

%------------------------------------------------

\flagverse{1.}
Jag vill börja gasqua, var fan är min flaska?\\*
Vem i helvete stal min butelj?\\*
Skall törsten mig tvinga, en TT börja svinga\\*
Nej, för fan bara blunda och svälj\\*
Vilken smörja, får jag spörja\\!

\flagverse{2.}
/:Vem för fan tror att jag är en älg?\\*
Till England vi rider och sedan vad det lider\\*
Träffar vi välan på någon pub\\*
Och där skall vi festa, blott dricka av det bästa\\*
Utav Whisky och portvin, jag tänker gå hårt in\\*
För att prova på rubb och stubb:/\\!

%------------------------------------------------

\poemauthorcenter{Bergsspexet, 1960}

%------------------------------------------------

\end{verse}

%----------------------------------------------------------------------------------------