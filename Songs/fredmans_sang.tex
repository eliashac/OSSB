%%%%%%%%%%%%%%%%%%%%%%%%%%%%%%%%%%%%%%%%%%%%%%%%%%%%%%%%%%%%%%%%%%%%%%%%%%%%%%%%
% Fredmans Sång
%%%%%%%%%%%%%%%%%%%%%%%%%%%%%%%%%%%%%%%%%%%%%%%%%%%%%%%%%%%%%%%%%%%%%%%%%%%%%%%%

\settowidth{\versewidth}{Och du, du Yngling, lyd min lag,}

\poemtitle{Fredmans Sång N:o 21}

%------------------------------------------------

\begin{verse}[\versewidth]

%------------------------------------------------

\flagverse{1.}
Så lunka vi så småningom\\*
Från Bacchi buller och tumult,\\*
När döden ropar, Granne kom,\\*
Ditt timglas är nu fullt.\\*
Du Gubbe fäll din krycka ner,\\*
Och du, du Yngling, lyd min lag,\\*
Den skönsta Nymph som åt dig ler\\*
Inunder armen tag.\\!

\flagverse{Refr.}
Tycker du at grafven är för djup,\\*
Nå välan så tag dig då en sup,\\*
Tag dig sen dito en, dito två, dito tre,\\*
Så dör du nöjdare.\\!

\flagverse{2.}
Du vid din remmare och präss,\\*
Rödbrusig och med hatt på sned,\\*
Snart skrider fram din likprocess\\*
I några svarta led;\\*
Och du som pratar där så stort,\\*
Med band och stjernor på din rock,\\*
Ren snickarn kistan färdig gjort,\\*
Och hyflar på des lock.\\!

\flagverse{Refr.}
Tycker du at grafven...\\!

\flagverse{3.}
Men du som till din återfärd,\\*
Ifrån det du till bordet gick,\\*
Ej klingat för din raska värd,\\*
Fastän han ropar: Drick!\\*
Drif sådan gäst från mat och vin,\\*
Kör honom med sitt anhang ut,\\*
Och sen med en ovänlig min,\\*
Ryck remmarn ur hans trut.\\!

\flagverse{Refr.}
Tycker du at grafven...\\!

\flagverse{3.}
Säg är du nöjd? min granne säg,\\*
Så prisa värden nu til slut;\\*
Om vi ha en och samma väg,\\*
Så följoms åt; drick ut.\\*
Men först med vinet rödt och hvitt\\*
För vår Värdinna bugom oss,\\*
Och halkom sen i grafven fritt,\\*
Vid aftonstjernans bloss.\\!

\flagverse{Refr.}
Tycker du at grafven...\\!

%------------------------------------------------

\end{verse}

%----------------------------------------------------------------------------------------