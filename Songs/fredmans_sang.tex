%%%%%%%%%%%%%%%%%%%%%%%%%%%%%%%%%%%%%%%%%%%%%%%%%%%%%%%%%%%%%%%%%%%%%%%%%%%%%%%%
% Fredmans Sång
%%%%%%%%%%%%%%%%%%%%%%%%%%%%%%%%%%%%%%%%%%%%%%%%%%%%%%%%%%%%%%%%%%%%%%%%%%%%%%%%

\settowidth{\versewidth}{Och du, du Yngling, lyd min lag,}

\poemtitle{Fredmans Sång N:o 21}

%------------------------------------------------

\begin{verse}[\versewidth]

%------------------------------------------------

\flagverse{1.}
Så lunka vi så småningom\\*
Från Bacchi buller och tumult,\\*
När döden ropar "Granne kom\\*
ditt timglas är nu fullt".\\*
Du Gubbe fäll din krycka ner\\*
Och du, du Yngling, lyd min lag,\\*
den skönsta Nymph som åt dig ler\\*
inunder armen tag.\\!

\flagverse{Refr.}
Tycker du att grafven är för djup?\\*
Nå välan så tag dig då en sup.\\*
Tag dig sen dito en, dito två, dito tre.\\*
Så dör du nöjdare.\\!

\flagverse{2.}
Du vid din remmare och präss,\\*
rödbrusig och med hatt på sned,\\*
snart skrider fram din likprocess\\*
i några svarta led.\\*
Och du som pratar där så stort\\*
Med band och stjernor på din rock,\\*
Ren snickarn kistan färdig gjort\\*
och hyflar på des lock.\\!

\flagverse{Refr.}
Tycker du att grafven...\\!

\flagverse{3.}
Men du som med en trumpen min\\*
bland riglar, galler, järn och lås,\\*
dig vilar på ditt penningskrin\\*
inom din stängda bås.\\*
Och du som svartsjuk slår i kras\\*
buteljer, speglar och pokal,\\*
bjud nu god natt, drick ur ditt glas\\*
och hälsa din rival.\\!

\flagverse{Refr.}
Tycker du att grafven...\\!

\flagverse{4.}
Och du som under titlars klang\\*
din tiggarstav förgyllt vart år\\*
som knappast har, med all din rang,\\*
en skilling till din bår.\\*
och du som ilsken, feg och lat\\*
fördömer vaggan som dig välvt\\*
och ändå dagligt är plakat\\*
till glasets sista hälft.\\!

\flagverse{Refr.}
Tycker du att grafven...\\!

\flagverse{5.}
Du som vid Martis fältbasun\\*
i blodig skjorta sträckt ditt steg\\*
och du som tumlar i paulun\\*
i Chloris armar feg.\\*
Och du som med din gyllne bok\\*
vid templets genljud reser dig\\*
som rister huvud lärd och klok\\*
och för mot avgrund krig\\!

\flagverse{Refr.}
Tycker du att grafven...\\!

\flagverse{6.}
Men du som med en ärlig min\\*
plär dina vänner häda jämt\\*
och dem förtalar vid ditt vin\\*
och det liksom på skämt.\\*
Och du som ej försvarar dem\\* 
fastän ur deras flaskor du,\\*
du väl kan slicka dina fem,\\*
vad svarar du väl nu?\\!

\flagverse{Refr.}
Tycker du att grafven...\\!


\flagverse{7.}
Men du som till din återfärd,\\*
ifrån det du till bordet gick,\\*
ej klingat för din raska värd,\\*
fastän han ropar: Drick!\\*
Drif sådan gäst från mat och vin,\\*
kör honom med sitt anhang ut,\\*
och sen med en ovänlig min,\\*
ryck remmarn ur hans trut!\\!

\flagverse{Refr.}
Tycker du att grafven...\\!

\flagverse{8.}
Säg är du nöjd? min granne säg,\\*
så prisa värden nu til slut;\\*
om vi ha en och samma väg,\\*
så följoms åt; drick ut.\\*
Men först med vinet rödt och hvitt\\*
för vår värdinna bugom oss,\\*
och halkom sen i grafven fritt,\\*
vid aftonstjernans bloss.\\!

\flagverse{Refr.}
Tycker du att grafven...\\!

%------------------------------------------------

\end{verse}

%----------------------------------------------------------------------------------------