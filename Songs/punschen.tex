%%%%%%%%%%%%%%%%%%%%%%%%%%%%%%%%%%%%%%%%%%%%%%%%%%%%%%%%%%%%%%%%%%%%%%%%%%%%%%%%
% Punschens Lov
%%%%%%%%%%%%%%%%%%%%%%%%%%%%%%%%%%%%%%%%%%%%%%%%%%%%%%%%%%%%%%%%%%%%%%%%%%%%%%%%
\settowidth{\versewidth}{som inget kan fördriva.}
\poemtitle{Punschens Lov}

%------------------------------------------------

\begin{verse}[\versewidth]

\flagverse{}
\emph{Mel. Rövarna i Kamomilla stad (Rövarvisan)}\\!

\flagverse{1.}
Punschen är och Punschen var\\*
och Punschen skall förbliva,\\*
en lidelse vi alla har\\*
som inget kan fördriva.\\*
Ja, Punschen tinar upp så väl\\*
och svalkar både kropp och själ.\\*
Den botar begären,\\*
och lindrar besvären.\\*
Ja, Punschen, den gör både gott och väl.\\!

%------------------------------------------------

\poemauthorcenter{Ur KTH-kårspexet Sven Hedin, 1987}

%------------------------------------------------

\end{verse}

%----------------------------------------------------------------------------------------