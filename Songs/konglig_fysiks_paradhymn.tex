%%%%%%%%%%%%%%%%%%%%%%%%%%%%%%%%%%%%%%%%%%%%%%%%%%%%%%%%%%%%%%%%%%%%%%%%%%%%%%%%
% Konglig Fysiks Paradhymn
%%%%%%%%%%%%%%%%%%%%%%%%%%%%%%%%%%%%%%%%%%%%%%%%%%%%%%%%%%%%%%%%%%%%%%%%%%%%%%%%
\settowidth{\versewidth}{Frukta ej ty hjälpen är just här}
\poemtitle{Konglig Fysiks Paradhymn (Fysik)}

%------------------------------------------------

\begin{verse}[\versewidth]

\flagverse{}
\emph{Mel. Katyuscha}\\!

\flagverse{1.}
Här på festen stiger åter glammet\\*
Sången börjar, tentan bortglömd är\\*
Lyss min strupe Du plågas utav dammet\\*
Frukta ej ty hjälpen är just här\\*
ibland oss\\*
Höj pokalen, dess flöde känns som sammet\\*
Drick till det som Bacchi vapen lär.\\!

%------------------------------------------------

\flagverse{2.}
Känn, Osquarulda, känn hur blodet hettar\\*
Osqarulda orsak till det är\\*
Timmar skrider och dygdens bojor lättar\\*
Fest och glädje kärleksflamman när\\*
men minns att\\*
Blott ej synen en hungrig kärlek mättar\\*
Drick till det som Venus' vapen lär.\\!

%------------------------------------------------

\flagverse{3.}
FYSIKER, gasqueropen de har skallat\\*
Likt musik från någon högre sfär\\*
Tentans piska för länge har oss vallat\\*
Trotsa den och studiernas misär\\*
med lärdom\\*
Från de makter som ytterst har oss kallat\\*
Bacchus, Venus värdar hos oss är\\*
och vänner\\*
Bacchi nektar ej Venus' flamma släcker \\*
SKÅL för det Fysiks skyddsgudar lär. \\!

%------------------------------------------------

\poemauthorcenter{Dum-Dum 1977}

%------------------------------------------------

\end{verse}

%----------------------------------------------------------------------------------------