%%%%%%%%%%%%%%%%%%%%%%%%%%%%%%%%%%%%%%%%%%%%%%%%%%%%%%%%%%%%%%%%%%%%%%%%%%%%%%%%
% Punschfinalen
%%%%%%%%%%%%%%%%%%%%%%%%%%%%%%%%%%%%%%%%%%%%%%%%%%%%%%%%%%%%%%%%%%%%%%%%%%%%%%%%

\settowidth{\versewidth}{I kväll har Napoleon gjort England den äran}

\poemtitle{Punschfinalen}

%------------------------------------------------

\begin{verse}[\versewidth]

\flagverse{}
\emph{Mel. Ryska nationalsången}\\!

%------------------------------------------------

\flagverse{1.}
I kväll har Napoleon gjort England den äran\\*
att komma på middag till dess ambassad.\\*
Men nu har ni sett till allas förfäran\\*
att blott tomma flaskor står på parad.\\!

%------------------------------------------------

\flagverse{Refr.}
Så går vi ner, ner till stadens hamnkvarter\\*
och vid kajen Victory vi ser.\\*
Vi går ombord och där väntar ett Nachspiel på oss.\\*
Vi går till bords både fiende och vän\\*
och våra gräl dom sparar vi till sen.\\*
Vi tar en matbit, en sup, nej flera förståss.\\!

%------------------------------------------------

\flagverse{2.}
Idag har lord Nelson krökat i Neapel\\*
Han raglar omkring med armen i en slips.\\*
Men snart står han uppsträckt i London på en stapel\\*
Vi undrar om någon i groggen har lagt gips.\\!

%------------------------------------------------

\flagverse{Refr.}
Så går vi ner, ner till stadens hamnkvarter...\\!

%------------------------------------------------

\poemauthorcenter{Bergsspexet Lord Nelson, 1975}

%------------------------------------------------

\end{verse}

%----------------------------------------------------------------------------------------