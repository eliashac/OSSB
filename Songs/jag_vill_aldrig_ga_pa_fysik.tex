%%%%%%%%%%%%%%%%%%%%%%%%%%%%%%%%%%%%%%%%%%%%%%%%%%%%%%%%%%%%%%%%%%%%%%%%%%%%%%%%
% Jag vill aldrig gå på Fysik
%%%%%%%%%%%%%%%%%%%%%%%%%%%%%%%%%%%%%%%%%%%%%%%%%%%%%%%%%%%%%%%%%%%%%%%%%%%%%%%%
\settowidth{\versewidth}{Vi kan bara räkna kvarkar}
\poemtitle{Jag vill aldrig gå på fysik}

%------------------------------------------------

\begin{verse}[\versewidth]

\flagverse{}
\emph{Mel. O, hur saligt att få vandra}\\!

\flagverse{1.}
Jag vill aldrig gå på Fysik\\*
Aldrig tenta termometerdynamik.\\*
Jag vill inte höra syntmusik\\*
Inte festa som ett jävla matte-geek\\*
Ni ser ut som televerket\\*
I vår jätte fula overål (alt. overall)\\*
Ni kan bara räkna kvarkar\\*
Nu hyllar vi data med en skål\\!

%------------------------------------------------

\flagverse{Refr.}
/:Fysik é torrt - Jag vill ju bort ...\\*
Fysik é torrt - Jag vill ju bort ...\\*
Fysik é torrt - Jag vill ju bort ...\\*
Fysik é torrt...:/\\!

%------------------------------------------------

\flagverse{2.}
Einsteins pojkar är vi alla\\*
Handels flickor kan vi aldrig få\\*
Går och tror att vi har ballar\\*
Det får bli på egen hand om det ska gå\\*
Nu ska Sing-Sing rivas,\\*
Arkitekt är med på datas lag\\*
Televerket skall fördrivas\\*
Uppå Konsulatets ljuva dommedag!\\!

%------------------------------------------------

\flagverse{Refr.}
/:Åh, nubbe drag - på denna dag ...:/\\!

%------------------------------------------------

\poemauthorcenter{DKM hösten 2000}

%------------------------------------------------

\end{verse}

%----------------------------------------------------------------------------------------