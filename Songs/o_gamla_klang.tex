%%%%%%%%%%%%%%%%%%%%%%%%%%%%%%%%%%%%%%%%%%%%%%%%%%%%%%%%%%%%%%%%%%%%%%%%%%%%%%%%
% O, gamla klang- och jubeltid
%%%%%%%%%%%%%%%%%%%%%%%%%%%%%%%%%%%%%%%%%%%%%%%%%%%%%%%%%%%%%%%%%%%%%%%%%%%%%%%%

\settowidth{\versewidth}{vår sång blir stum, vårt glam förstämt;}

\poemtitle{O, gamla klang- och jubeltid}

%------------------------------------------------

\begin{verse}[\versewidth]

\flagverse{1.}
O, gamla klang- och jubeltid,\\*
ditt minne skall förbliva\\*
och än åt livets bistra strid\\*
ett rosigt skimmer giva!\\*
Snart tystnar allt vårt yra skämt,\\*
vår sång blir stum, vårt glam förstämt;\\*
o, jerum, jerum, jerum,\\*
o, quae mutatio rerum!\\!

%------------------------------------------------

\flagverse{2.}
Var äro de, som kunde allt,\\*
blott ej sin ära svika,\\*
som voro män av äkta halt\\*
och världens herrar lika?\\*
De drogo bort från vin och sång\\*
till vardagslivets tråk och tvång;\\*
o, jerum, jerum, jerum,\\*
o, quae mutatio rerum!\\!

%------------------------------------------------

\flagverse{3.}
Den ene vetenskap och vett\\*
in i scholares mänger,\\*
den andre i sitt anlets svett\\*
på paragrafer vränger,\\*
en plåstrar själen som är skral,\\*
en lappar hop dess trasiga fodral;\\*
o, jerum, jerum, jerum,\\*
o, quae mutatio rerum!\\!

%------------------------------------------------

\flagverse{4.}
En tämjer forsens vilda fall,\\*
en annan ger oss papper,\\*
en idkar maskinistens kall,\\*
en mästrar volt så tapper,\\*
en ritar hus, en mäter mark,\\*
en blandar hop mixtur så stark;\\*
o, jerum, jerum, jerum,\\*
o, quae mutatio rerum!\\!

%------------------------------------------------

\flagverse{5.}
Men hjärtat i en sann student\\*
kan ingen tid förfrysa,\\*
den glädjeeld som där han tänt,\\*
hans hela liv skall lysa.\\*
Det gamla skalet brustit har,\\*
men kärnan finnes frisk dock kvar,\\*
och vad han än må mista\\*
den skall dock aldrig brista!\\!

%------------------------------------------------

\flagverse{6.}
Så sluten, bröder, fast vår krets\\*
till glädjens värn och ära!\\*
Trots allt vi tryggt och väl tillfreds\\*
vår vänskap trohet svära.\\*
Lyft bägar'n högt, och klinga, vän!\\*
De gamla gudar leva än\\*
bland skålar och pokaler,\\*
bland skålar och pokaler!\\!

%------------------------------------------------

\poemauthorcenter{E Höfling, 1825}

%------------------------------------------------

\end{verse}

%----------------------------------------------------------------------------------------